%% Modelo de Relatório Fapesp

%% Tipo de documento
\documentclass[
    12pt,
    openright, 
    a4paper,
    english,
    brazil,
    article
]{abntex2}

%% Pacotes utilizados
\usepackage{lmodern}
\usepackage[T1]{fontenc}
\usepackage[utf8]{inputenc}
\usepackage{amsthm,amssymb,amsmath}
\usepackage{graphicx}
\usepackage{microtype}
\usepackage{float}
\usepackage[left=3.3cm,right=3.3cm]{geometry}
\usepackage[brazil]{babel}
\usepackage{abntex2cite} % Citações ABNT
    \citebrackets[]
\usepackage{indentfirst}
\usepackage{calrsfs}

%% Redefinições e novos comandos
\renewcommand{\baselinestretch}{1.5}
\newcommand{\doublesignature}[3][Júlia Wotzasek Pereira]{
    \parbox{\textwidth}{
        \vspace{2cm}
        \parbox{7cm}{
            \centering
            \rule{6cm}{1pt}\\
            #1
        }
        \hfill
        \parbox{7cm}{
            \centering
            \rule{6cm}{1pt}\\
            #2
        }
    }
}
\newcommand{\eqdef}{\overset{\mathrm{def}}{=\joinrel=}}

%% Definições matemáticas
\newtheorem{thm}{Teorema}[section]
\newtheorem{lemma}[thm]{Lema}
\newtheorem{prop}[thm]{Proposição}
\newtheorem{corl}[thm]{Corolário}
\newtheorem{conj}[thm]{Conjectura}
\theoremstyle{definition}
\newtheorem{defn}[thm]{Definição}
\newtheorem{exmp}[thm]{Exemplo}
\newtheorem{obs}[thm]{Observação}

%% Definição do Título do Projeto
\title{Estruturas Algébricas em Heranças Genéticas}

%% Início do Documento
\begin{document}
%% Parte A - Capa e Resumo
\clearpage
\pagestyle{empty}
\begin{center}
    \Large{\textbf{Relatório de Progresso de Bolsa de Iniciação Científica}}
\end{center}

\vspace{7em}
\noindent\hrulefill 

\noindent \textbf{Título:} Estruturas algébricas em heranças genéticas
\\ \textbf{Aluna:} Júlia Wotzasek Pereira
\\ \textbf{Orientador:} Thiago Castilho de Mello
\\ \textbf{Instituição:} Instituto de Ciência e Tecnologia -\\ \indent - Universidade Federal de São Paulo - UNIFESP
\\ \textbf{Número de Processo:} 2018/10312-3
\\ \textbf{Vigência:} 01/08/2018 a 31/12/2019
\\ \textbf{Período de Referência:} 01/08/2019 a 31/12/2019

\noindent\hrulefill

\vspace{9em}
\begin{center}
\doublesignature{Thiago Castilho de Mello}{}
    
\end{center}

\clearpage
\pagestyle{plain}
\pagenumbering{arabic}
\section{Introdução}

\subsection{Resumo do Plano de Trabalho}
O projeto consiste no estudo dos seguintes tópicos:
\begin{enumerate}
    \item ...
\end{enumerate}

\subsection{Resumo das atividades realizadas no período}

Conforme previsto no plano de trabalho, ...

%% Parte B - Desenvolvimento
\clearpage
% Definições e Propriedades Básicas de EACP
\section{Definições e Propriedades Básicas de EACP}



%% Parte C - Conclusão
\clearpage
\section{Conclusão}



%% Parte D - Referências
\newpage
\bibliography{referencias}
\end{document}